\documentclass{article}
\usepackage{graphicx} % Required for inserting images
\usepackage{listings}
\usepackage{hyperref}
\usepackage{tikz}
\usepackage{float}

\title{ARQUITETURA E DESENVOLVIMENTO DE MICROSSERVICOS}
\author{Henrique Rodrigues - 45224}
\date{Maio 2025}

\begin{document}

\maketitle

\section{Introdução}
Este documento descreve a arquitetura e implementação de um sistema de microsserviços desenvolvido utilizando tecnologias modernas como Fastify, PostgreSQL, Docker e Kubernetes. O sistema é composto por múltiplos serviços independentes que se comunicam entre si através de uma camada de orquestração, proporcionando escalabilidade e manutenibilidade.

\section{Arquitetura do Sistema}
\subsection{Visão Geral}
O sistema segue uma arquitetura de microsserviços com os seguintes componentes principais:
\begin{itemize}
    \item \textbf{Cliente}: Interface web gerada automaticamente pelo Swagger UI
    \item \textbf{Orquestrador}: Serviço central que roteia requisições e gerencia comunicação
    \item \textbf{Serviços}: Auth, Users e Tasks
    \item \textbf{Banco de Dados}: PostgreSQL
\end{itemize}

\subsection{Fluxo de Comunicação}
O fluxo de comunicação segue o padrão:
\begin{verbatim}
Cliente -> Orquestrador -> Serviço Específico -> Banco de Dados
\end{verbatim}

\section{Serviços}
\subsection{Orquestrador (Porta 3000)}
O orquestrador atua como gateway da API, gerenciando:
\begin{itemize}
    \item Roteamento de requisições
    \item Autenticação via JWT
    \item Documentação Swagger UI
    \item Health checks dos serviços
\end{itemize}

\subsection{Serviço de Autenticação (Porta 3001)}
Responsável por:
\begin{itemize}
    \item Registro de usuários
    \item Login
    \item Verificação de tokens JWT
\end{itemize}

\subsection{Serviço de Usuários (Porta 3002)}
Gerencia:
\begin{itemize}
    \item Atualização de dados do usuário
    \item Deleção de conta
    \item Consulta de informações do usuário
\end{itemize}

\subsection{Serviço de Tarefas (Porta 3003)}
Oferece operações CRUD para tarefas:
\begin{itemize}
    \item Criação de tarefas
    \item Atualização de tarefas
    \item Listagem de tarefas
    \item Deleção de tarefas
\end{itemize}

\section{Tecnologias Utilizadas}
\subsection{Backend}
\begin{itemize}
    \item \textbf{Fastify}: Framework Node.js de alta performance
    \item \textbf{PostgreSQL}: Banco de dados relacional
    \item \textbf{JWT}: Autenticação stateless
\end{itemize}

\subsection{Documentação}
\begin{itemize}
    \item \textbf{Swagger UI}: Interface web automática
    \item \textbf{OpenAPI}: Especificação da API
\end{itemize}

\subsection{Infraestrutura}
\begin{itemize}
    \item \textbf{Docker}: Containerização dos serviços
    \item \textbf{Kubernetes}: Orquestração de containers
\end{itemize}

\section{Configuração Kubernetes}
O sistema utiliza dois arquivos principais de configuração Kubernetes:

\subsection{k8s-services.yaml}
Define os deployments e services para:
\begin{itemize}
    \item Orquestrador
    \item Serviço de Autenticação
    \item Serviço de Usuários
    \item Serviço de Tarefas
\end{itemize}

\subsection{k8s-postgres.yaml}
Configura:
\begin{itemize}
    \item Deployment do PostgreSQL usando imagem alpine
    \item ConfigMap com script de inicialização (schema.sql)
    \item Secret para senha do banco
    \item Service para acesso ao banco (ClusterIP)
\end{itemize}

\textbf{Nota}: Por simplicidade, o banco de dados não utiliza volumes persistentes, sendo recriado a cada reinicialização do pod. Em um ambiente de produção, seria necessário adicionar um PersistentVolume para garantir a persistência dos dados.

\section{Scripts de Gerenciamento}
\subsection{Setup (k8s-setup.sh)}
Script responsável por:
\begin{itemize}
    \item Criar namespace Kubernetes
    \item Aplicar configurações do PostgreSQL
    \item Aplicar configurações dos serviços
    \item Aguardar todos os pods estarem prontos
\end{itemize}

\subsection{Cleanup (cleanup.sh)}
Script para limpeza do ambiente:
\begin{itemize}
    \item Remover todos os deployments
    \item Remover todos os services
    \item Remover persistent volumes
    \item Deletar namespace
\end{itemize}

\section{Banco de Dados}
O schema do PostgreSQL (schema.sql) define:
\begin{itemize}
    \item Tabela de usuários com campos para autenticação
    \item Tabela de tarefas com relacionamento com usuários
    \item Extensão UUID para identificadores únicos
\end{itemize}

\section{Instalação e Execução}
O script install-all.sh automatiza:
\begin{itemize}
    \item Instalação de dependências npm em todos os serviços
    \item Verificação de diretórios com package.json
    \item Execução do npm install em paralelo
\end{itemize}

\end{document}
